\documentclass[11pt, a4paper]{article}

\usepackage[left=2cm,text={17cm, 24cm},top=3cm]{geometry}
\usepackage[utf8]{inputenc}
\usepackage[czech]{babel}
\usepackage{times}
\usepackage{marvosym}

\author{Jan Kubica}

\providecommand{\uv}[1]{\quotedblbase #1\textquotedblleft}

\begin{document}
	\begin{titlepage}
		\begin{center}
			\textsc{\Huge Vysoké učení technibcké v~Brně} \\[8pt]
			\textsc{\huge Fakulta informačních technologií} \\[16pt]
			\vspace{\stretch{0.382}}
			{\LARGE Typografie a publikování -- 4. projekt} \\[6pt]
			{\Huge Bibliografické citace}
			\vspace{\stretch{0.618}}
		\end{center}
		{\Large \today \hfill Jan Kubica}
	\end{titlepage}


\section{Historie a vývoj písma}

Písmo, jak jej užíváme dnes, sahá až do historie okolo 5000 let př.\,n.\,l. Mezi nejstarší se řadí tzv. \textbf{piktogramy}, nebo-li znaky znázorňující popisované věci obrazově. Postupem času dospěla každá z~kultur do různého stupně vývoje. V~současnosti můžeme písmo rozdělit do 4 základních skupin: \emph{latinkové}, \emph{nelatinkové pravosměrné}, \emph{nelatinkové
levosměrné} a \emph{ostatní}.
\cite{wiki:pismo}\cite{znak_sady}

\section{Rodina písma}
Rodinou písma označujeme skupinu několika řezů vytvořených z~jednoho typu písma. Řez písma je tak určitá kresebná varianta základního typu písma, kterou používáme pro zvýrazňování jednotlivých slov nebo delšího textu. Jedná se např. o~\textbf{tučné písmo} (v~textovém editoru \LaTeX\ pomocí příkazu \verb|\textbf{}|) nebo \linebreak \emph{kurzívu} (\verb|\emph{}|).\cite{manual}\cite{tex_companion}

\subsection{Známé rodiny písem}
Jedny z~světově nejpoužívanějších rodin písem jsou \texttt{Times New Roman}, \texttt{Arial} a \texttt{Helvetica}.\linebreak \texttt{Times New Roman} je obecně základním \emph{serifovým} (patkovým) písmem operačního systému Windows \linebreak od~Microsoftu. \texttt{Helvetica} a \texttt{Arial} jsou pak typickými \emph{sans-serifovými} (bezpatkovými) písmy.\cite{typomil}

\subsection{Další populární písma}
Jak zmiňuje časopis Type, za povšimnutí stojí i písmo nazvané \textbf{Windsor}. Toto písmo původně z~Anglie z~roku 1905 se hodí především jako tvarově i barevně velmi výrazný titulek. Je zajímavé zejména díky své kresební zvláštnosti, kdy k~osobitým prvkům patří výrazně sešikmené serify u~některých verzálek.\cite{font_2012}

\subsection{Současná tvorba}
Jedním z~nejúspěšnějších projektů v~České republice jsou písma Vojtěcha Říhy, který za svůj soubor písem získal ocenění v~prestižní soutěži \uv{Graduation Projects 2015}. Vojtěch se tak rozhodl založit vlastní písmolijnu \textbf{Superior Type} a v~současnosti poskytuje 3 písma: \emph{Hrot}, \emph{Kunda Book} a \emph{Vegan Sans}.\cite{font_2016}

\section{Výhody a nevýhody počítačového písma}

Počítačové písmo se od tiskového písma předchozích staletí v~mnohém liší. Každý znak je dnes pouze matematickým zápisem jedniček a nul a písmo tak ztratilo náhodné chybičky v~kresbě, která činila tištěné texty lidštějšími a přívětivějšími. Na druhou stranu v~době klasického knihtisku byly používány na každou velikost písma předem určené a stylizované raznice. Větší písma měla jiný vzhled, aby proporcionálně působila přirozeně a písmo tak bylo přesně předurčeno pro své použití. Počítačová technologie tuto závislost zcela odstranila. Typograf tudíž najednou může použít plakátové písmo na vizitce a knižní antikvu na billboardu.\cite{zivy_font}


\pagebreak

\bibliographystyle{czechiso}
\bibliography{proj4}


\end{document}